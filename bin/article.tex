\documentclass[12pt]{article}
\usepackage{crimson}
\usepackage[T1]{fontenc}
\usepackage[french]{babel}
\usepackage{geometry}
\geometry{a4paper, margin=1in}

\renewcommand{\tiny}{\normalsize}
\renewcommand{\footnotesize}{\normalsize}
\renewcommand{\small}{\normalsize}
\renewcommand{\large}{\normalsize}
\renewcommand{\Large}{\normalsize}
\renewcommand{\LARGE}{\normalsize}
\renewcommand{\huge}{\normalsize}
\renewcommand{\Huge}{\normalsize}

\begin{document}

\begin{flushright}
  \textbf{Ricardo Stefan Devai Melo}
  \linebreak
  \textbf{Brest, France}
  \linebreak
  \textbf{03/08/2019}
  \linebreak
\end{flushright}

\noindent{\textbf{\uppercase{Apologie pour l'histoire ou Métier d'historien}}}

\section*{Caractères généraux de l'observation historique}
[auteur]

Le chapitre en question a été écrit par Marc Bloch, un historien français né le 6 juillet 1886. Issu d'une famille juive et d'un père professeur d'histoire ancienne à l'université de Lyon, Bloch a fait ses études en histoire et géographie [1]. En 1929, avec Lucien Febvre, il a fondé l'École des Annales, un courant historiographique qui a souligné l'importance des thèmes de caractère social et économique.

[contexte de l'ouvre]

Quelques années plus tard, en 1940, la Seconde Guerre Mondiale aura lieu, ce fait va provoquer un grand changement de ses conditions de travail, . C'est donc dans ce contexte que Bloch commença à rédiger \emph{Apologie pour l'histoire ou Métier d'historien}. En 1944, Marc Bloch est capturé et fusillé par l'Allemagne nazi, laissant son travail incomplet.

[synthèse de l'ouvre]

[analyse de l'ouvre]

\end{document}
