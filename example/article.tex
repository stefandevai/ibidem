\documentclass{article}
\usepackage{hyperref}
\usepackage[T1]{fontenc}

\begin{document}



\begin{flushright}
\textbf{DEVAI MELO Ricardo Stefan}
\linebreak
\textbf{Sorbonne Université - L1 Histoire}
\linebreak
\textbf{27/02/2020}
\linebreak
\end{flushright}

\section*{L'évangélisation des Indiens d'Amérique (XVIe - XVIIe siècles)}

Dans la deuxième moitié du XVIIe siècle, John Eliot, un puritain anglais, traduit lui même la bible au wôpanâak, une langue indienne qui n'avait pas d'expression écrite. Il habite avec des indiens et organise une communauté où ils cohabitent avec des puritains. Cette prétendue harmonie entre indiens et puritains cache une réalité complexe de supériorité religieuse du côté puritain et de soumission colonial du côté indien c[eliot2:32].

Les indiens d'Amérique sont l'ensemble de peuples natifs du continent américain. Pour notre analyse, on se concentre surtout dans les indiens d'Amérique du nord qui ont vécu le procès de colonisation britannique. Parmi ces peuples on trouve une large gamme de langues et de cultures différentes qui réagissent de diverses manières devant le contact avec les européens : certes cherchent une alliance avec eux, autres trouvent le conflit et autres s'enfuient simplement. Malgré ces différences, les arrivants les voient, dans la plupart des cas, comme des peuples primitifs et païens ceux qui doivent être libérés et convertis au christianisme. À ce procès de conversion on appelle \emph{évangélisation} ; c'est donc l'essai d'imposer les doctrines chrétiennes au détriment des religions natives.

On cherche comprendre ces termes dans le contexte de la fin du XVIe et le XVIIe siècles dans le territoire occupé par les colons britanniques. Si dans la plupart du XVIe siècle les essaies de évangélisation des indiens sont faites par des ordres catholiques espagnoles et françaises, c'est à la fin du XVIe siècle qu'on voit les premiers contacts entre indiens et européens. Les deux premiers essaies de colonisation britannique échouent et c'est juste en 1607 que la Compagnie de Londres réussit à établir Jamestown comme une colonie permanente. Dans ce moment les colons britanniques ont la stabilité nécessaire pour se concentrer de mode plus organisé dans le procès d'évangélisation.

On verra donc dans quelle mesure les arrivants européens en Amérique méprisaient-ils les religions natives et comprennent l'évangélisation comme un acte qui peut sauver les indiens d'un état primitif de leur civilisation.

Dans un premier moment, il est nécessaire de comprendre les caractéristiques particulières des indiens du nord-ouest américain et des évangélisateurs ; Ensuite, il sera exposé l'analyse du procès d'évangélisation comme l'expression d'une thèse et une antithèse ; enfin, on pourra comprendre ainsi le produit de ces rencontres religieux dans la société colonial du XVIIe siècle.

Les indiens en Amérique ne sont pas un groupe homogène, on y trouve plusieurs familles de langues, formes d'organisation social et d'expression spirituelle. Pourtant, on se focalise sur les indiens du nord-est d'Amérique car ils sont les premiers indiens qui ont contacté les arrivants britanniques. On peut encore les diviser en deux grands groupes : les algonquiens et les iroquois c[britt]. La plupart des groupes sont des nomades, mais il y a aussi des sociétés agricoles, comme c'est le cas des Powhatan, l'un des premiers tribus à être en contact avec la colonie de Jamestown. Leur système de croyances existe autour d'un shaman, une autorité religieuse, qui est capable de soigner personnes et comprendre le monde à partir d'éléments naturels c[poter:113].

Les européens qui arrivent sont en grande partie des puritains britanniques qui s'enfuient de la persécution religieuse dans les Îles Britanniques. Ils se déménagent définitivement à Amérique avec le but de fonder une société où leur liberté religieuse soit assurée. Le puritanisme est une réaction aux influences catholiques chez les anglicans, les puritains cherchent donc \emph{purifier} l'anglicanisme et s'approchent ainsi des propositions calvinistes, tel comme la notion de prédestination pour le Salut c[britt2].

Le rencontre de ces deux groupes origine une collision en plusieurs aspects : parmi lesquels on peut citer le géographique, dans la dispute pour certains territoires ; le génétique, dans l'épidémie de variole qui tue une grande partie des indiens ; ou aussi l'économique, dans le commerce de différents produits originaires de chaque continent ; mais ce que nous intéresse dans cette analyse est la collision dans l'aspect historique-religieux : les pratiques religieuses sont complètement différentes entre les deux groupes et les essaies d'évangélisation explicitent ce contraste.

À propos des premiers moments de cet encontre, Harry Thomas Stock écrit dans le Journal Américain de Théologie : « \emph{The voyages of exploration during the last years of the sixteenth century almost invariably took account of the glorious prospect of saving the savage from heathenism} » c[stock:369]. Du côté puritain, les indiens sont des sauvages et l'évangélisation est l'act de les sauver d'un état sauvage, et dans ce sens, une forme de transmettre la vraie civilisation, leur civilisation, leur religion. L'évangélisation est vu par les puritains comme une attitude de charité vers une civilisation inférieure.

D'un autre côté, Amanda Porterfield du Centre d'Études de Religion et Culture Américaines écrit : « \emph{Algonquians and Iroquois embraced beliefs in witchcraft to explain the epidemics of smallpox, cholera, bubonic plague, diphtheria, typhoid, scarlet fever, measles, and influenza that accompanied colonization and decimated Native populations. They also invoked witchcraft to explain and cope with other changes generated by colonization as well, including both the deterioration of their precolonial cultures and their growing dependence on Western economies, governments, and technologies} » c[poter:104]. Depuis la perspective indienne, c'est difficile d'accepter comme vérité la religion de ceux qui les ont apporté des maladies et plusieurs souffrances. En certains cas, ils utilisent les catégories religieuses négatives pour définir les européens, comme dans ce cas, l'association de la colonisation à la sorcellerie.

Le résultat de ces collisions religieuses est parfois victorieux du côté britannique, comme dans l'exemple de John Eliot. Appelé Apôtre des Indiens, en 1650 il convaincre environ cinquante indiens de Massachusetts à se déménager à un village accompagnés de lui et d'une communauté de puritains. Dans ce village il apprendre les indiens à lire et à écrire et traduit la biblie à leur langue c[eliot:214]. Même si la décision de cohabiter avec les indiens peut sembler harmonieuse et humanitaire, Richard W. Cogley du Reed College affirme : « \emph{Eliot's belief that the Indians lacked "civility" betrayed his assumptions about the value of their culture. Like other colonists he thought that aboriginal culture represented the ultimate human degradation} » c[eliot:213]. D'ailleurs, la religion imposé aux peuples indiens est réinterprété est un espèce de hybridisme entre pratiques chrétiennes et natives est né c[fisher:115].

Les groupes indiens que ne veulent pas le conflit ni la soumission aux colons, optent pour migrer vers l'ouest d'Amérique et ainsi s'éloigner du centre de la colonisation britannique, ce qui permet la permanence de sa culture originale et la coexistence avec autres groupes indiens qui migrent de régions de l'Amérique espagnole.

À partir de cette analyse, c'est possible de constater que les indiens, ayant un modèle religieux très diffèrent de ceux qui les puritains veulent mettre comme base de leur société, trouvent ainsi un conflit inévitable avec les arrivants européens. Ces conflits finissent par intégrer, détruire ou par isoler certains tribus. En tout cas, l'effort puritain d'évangéliser ne cache jamais la vision paternaliste vers les indiens ; ils sont des sauvages et c'est l'envie de les sauver de cet état qui pousse le procès d'évangélisation. Pendant toute la période coloniale, et même après l'indépendance, les indiens d'Amérique ont fait l'effort de garder ses traditions religieuses ; c'est seulement en 1978, avec le \emph{American Indian Religious Freedom Act} que l'exercice de leur liberté religieuse est enfin assuré par la loi des États-Unis c[code].


\end{document}